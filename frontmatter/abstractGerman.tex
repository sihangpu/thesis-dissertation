%!TEX root = ../dissertation.tex
% the abstract
Im Gegensatz zu traditionellen kryptografischen Aufgaben, bei denen Kryptografie verwendet wird, um die Sicherheit und Integrit\"at von Kommunikation oder Speicherung zu gew\"ahrleisten und der Gegner typischerweise ein Außenstehender ist, der versucht, die Kommunikation zwischen Sender und Empf\"anger abzuhören, ist die Kryptografie, die in der datenschutzbewahrenden Berechnung (oder sicheren Berechnung) verwendet wird, darauf ausgelegt, die Privatsph\"are der Teilnehmer voreinander zu schützen.

Insbesondere erm\"oglicht die datenschutzbewahrende Berechnung es mehreren Parteien, gemeinsam eine Funktion zu berechnen, ohne ihre Eingaben zu offenbaren. Sie findet zahlreiche Anwendungen in verschiedenen Bereichen, einschließlich Finanzen, Gesundheitswesen und Datenanalyse. Sie erm\"oglicht eine Zusammenarbeit und Datenaustausch, ohne die Privatsph\"are sensibler Daten zu kompromittieren, was in der heutigen digitalen \"Ara immer wichtiger wird.

Obwohl datenschutzbewahrende Berechnung aufgrund ihrer starken Sicherheit und zahlreichen potenziellen Anwendungen in jüngster Zeit erhebliche Aufmerksamkeit erregt hat, bleibt ihre Effizienz ihre Achillesferse. Datenschutzbewahrende Protokolle erfordern deutlich h\"ohere Rechenkosten und Kommunikationsbandbreite im Vergleich zu Baseline-Protokollen (d.h. unsicheren Protokollen).

\begin{sloppypar}
Daher bleibt es eine spannende Aufgabe, M\"oglichkeiten zu finden, um den Overhead zu minimieren (sei es in Bezug auf Rechen- oder Kommunikationsleistung, asymptotisch oder konkret), w\"ahrend die Sicherheit auf eine angemessene Weise gew\"ahrleistet bleibt.
\end{sloppypar}

Diese Arbeit konzentriert sich auf die Verbesserung der Effizienz und Reduzierung der Kosten für Kommunikation und Berechnung für g\"angige datenschutzbewahrende Primitiven, einschließlich private Schnittmenge, vergesslicher Transfer und Stealth-Signaturen. Unser Hauptaugenmerk liegt auf der Optimierung der Leistung dieser Primitiven.