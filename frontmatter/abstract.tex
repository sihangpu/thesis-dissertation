%!TEX root = ../dissertation.tex
% the abstract

In traditional cryptographic applications, cryptographic mechanisms are employed to ensure the security and integrity of communication or storage. In these scenarios, the primary threat is usually an external adversary trying to intercept or tamper with the communication between two parties. On the other hand, in the context of privacy-preserving computation or secure computation, the cryptographic techniques are developed with a different goal in mind: to protect the privacy of the participants involved in a computation from each other.

Specifically, privacy-preserving computation allows multiple parties to jointly compute a function without revealing their inputs and it has numerous applications in various fields, including finance, healthcare, and data analysis. It allows for collaboration and data sharing without compromising the privacy of sensitive data, which is becoming increasingly important in today's digital age.

While privacy-preserving computation has gained significant attention in recent times due to its strong security and numerous potential applications, its efficiency remains its Achilles' heel. Privacy-preserving protocols require significantly higher computational overhead and bandwidth when compared to baseline (i.e., insecure) protocols.

Therefore, finding ways to minimize the overhead, whether it be in terms of computation or communication, asymptotically or concretely, while maintaining security in a reasonable manner remains an exciting problem to work on.

This thesis is centred around enhancing efficiency and reducing the costs of communication and computation for commonly used privacy-preserving primitives, including private set intersection, oblivious transfer, and stealth signatures. Our primary focus is on optimizing the performance of these primitives. 